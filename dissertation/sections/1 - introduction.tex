\section{Introduction}
\label{sec:section1} 
%To make a call to the introduction, put \ref{sec:Introduction} 

It is widely acknowledged that the search for love is a deeply relevant, personal, and complex social phenomenon, but in today's world, swipe-based dating applications (SBDA's) seem to only make it trickier. 
These platforms, exemplified by Tinder, Bumble, or Hinge, provide a gamified way of browsing through potential romantic partners by swiping through a stack of suggested candidates to indicate likes or dislikes for these, one profile at a time. 
In the search and matching literature, settings like these fall under the category of decentralized two-sided matching markets with online search \citep{kanoria2021facilitating}, emphasising three main characteristics. 
Firstly, that both sides of the market are comprised of decision-making agents undertaking a process of search. 
Secondly, that matches occur as outcomes of independently-determined search decisions, rather than through a centralised algorithm. Thirdly, that romantic suggestions are presented in an online manner to users, stressing the importance of sequential rationality within the search process given that it is not possible to interact with the same candidate twice. 
These apps differ widely from traditional dating sites where users are centrally and statically matched (such as match.com or eHarmony), but have come to dominate the modern love market, with Tinder alone boasting 75 million monthly active users and 9.6 million paid subscribers as of 2021 \citep{web:tinder_stats}.


From a theoretical standpoint, search within SBDA's encompasses many complexities that, due to the novelty of the platforms, have been sparsely studied in the economics literature. 
On one hand, platform-specific characteristics, such as swiping caps, asynchronicity, and the suggestion algorithms used impose non-trivial constraints to the way utility-maximising agents strategise their search process.
On the other hand, the general problem of search in a two-sided setting is interesting in and of itself, as a simple stage interaction (to swipe or not to swipe on a romantic suggestion) can become increasingly complex when repeated over an infinite horizon, admitting to problems such as intractable strategy spaces. 
Overall, the prevalent role of SBDA's in shaping modern romantic interactions, the theoretical complexities they induce, and their largely understudied nature motivates many different questions. 
Nevertheless, answering these requires a fundamental understanding of how users make decisions in these platforms: to put it simply, \textit{when should a utility-maximising user swipe right?} 


This paper will explore the above within the setting of a swipe-based dating platform, where agents heterogeneous preferences on both sides of the market search simultaneously for multiple romantic partners. I present an appropriate refinement for the mean-field equilibria of my model and approximate these at steady-state using computational methods. 
Crucially, I find that gender imbalances (which arise due to several exogenous factors that have been previously researched) can explain swipe rate disparities within the platform, a stylized fact that has been observed empirically in previous research. Furthermore, I model a possible intervention where the swiping cap ratio between sexes can be set in a socially-efficient manner in order to compensate for welfare losses due to gender imbalances.
This work presents three main contributions to existing literature on the topic. 
Firstly, it constitutes one of a handful of attempts to model market configurations arising within SBDA's, which is unsurprising due to the novelty of these platforms, but important given their current social relevance. 
Furthermore, this work distinguishes itself by considering the impact of swiping caps in the market, both as a constraint within the agent's search problem and a potential market correction mechanism.
Finally, this work provides a marginal side-contribution as a case study for the use of computational methods within game theory, a field that has traditionally emphasised pure mathematical analysis. To explore the above questions, this paper relies on a rigorously-formulated model, but also on numerical approximation algorithms and agent-based simulations, which can be used to compute equilibria and perform visually-intuitive comparative statics. 
As such, it shows that the two approaches, rather than being mutually exclusive, can be jointly employed to explore complicated questions, as computational methods enable quick explorations that can serve as a stepping stone towards formalising rigorous mathematical arguments.

The remainder of the paper is structured as follows. In \autoref{sec:section2}, I outline the theoretical framework for the model presented in this paper, and derive necessary conditions for both the system steady state and the agent best-response correspondences. In \autoref{sec:section3}, I present a refined definition for the steady state equilibrium of the model and perform computational comparative statics on several parameters in order to replicate stylized facts concerning the market configurations that arise in SBDA's. In \autoref{sec:section4}, I utilize agent-based simulation methods to analyse convergence and dynamics of my model, and present a discussion on socially-efficient budget interventions. Finally, in \autoref{sec:section5}, I present concluding remarks for my work and outline potential avenues for future work on the subject.

\subsection{Related Work}
The present work draws inspiration from two key branches of economics literature: that of search and matching, which studies the decision-making process of agents who seek, for example, a job, a business partner, or a spouse, and also that of mean field game theory, which has been employed to study complex dynamic games involving a large number of players. I discuss each of these in turn, and then contrast this work with the handful of papers that have focused on specifically analysing SBDA market configurations.

Within the search and matching literature there is an abundance of different theoretical models, amply summarised by \cite{chade2017sorting}, with several extensions studying a wide variety of different settings. As previously noted, three defining features of SBDA markets are decentralised matching, two-sidedness, and sequential interaction; one of the most prominent works on matching markets at this intersection is that of \cite{burdett1997marriage}, which studies a setting of uniform random search where agents receive marriage proposals from the other side of the market according to a continuous-time process, and must choose whether or not to accept these given the observable `pizazz' of the proposing agent. Several extensions followed this work, considering cases with noisy observations of `pizazz' \citep{chade2006matching}, idiosyncratic preferences \citep{burdett1998two}, directed search, and so on. Some papers have even studied the convergence of decentralised two-sided models like these onto the set of stable matchings, which can serve as a socially-efficient centralised benchmark \citep{adachi2003search}. This line of work served as a great inspiration for the different `flavours' of two-sided matching models that could best represent the SBDA market, and there are several key modelling aspects that I apply within this paper such as the endogenous flow-based approach used to unify search in both sides of the market. Despite this, the main difference between my model and others within this line of research occurs since agents in SBDA platforms search not only for spouses but also for casual relationships, thus demanding a framework that allows them to accumulate multiple matches (something that been largely understudied in preceding works due to the focus on marriage). 

On the other hand, mean field game theory focuses on dynamic games with a large number of agents, for which curses of dimensionality often arise, making solution concepts such as Markov Perfect Equilibria intractable \citep{maskin2001markov}. To deal with this, mean field models consider individual interactions with the \textit{aggregate system state}, ie. the distributions over states and strategies within the game, rather than interactions with all other players. This abstraction is coupled with the notion of a \textit{consistency check}, such that equilibrium arises when rational play given an aggregate state maintains this same state as a fixed point. The approach, first considered in the work of \cite{jovanovic1988anonymous}, greatly simplifies strategic settings with the aforementioned problem and has been successfully applied to settings such as network routing \citep{calderone2017markov}, auctions with learning \citep{iyer2014mean}. In this paper, we rely on mean-field considerations to abstract from considerations on observability; within SBDA's, the market-wide history and opponent state are unobservable to players, and thus traditional equilibrium concepts would demand beliefs over uncountable history spaces, and even beliefs over the beliefs other players may hold (a complication known in the literature as nested beliefs \citep{brandenburger1993hierarchies}). This yields two central problems: first, that equilibria become virtually impossible to compute, and, by extension, that the model assumes an unreasonable level of rationality on behalf of agents, especially given that these rarely interact with the same individual twice amongst millions of other users. Thus, by considering interactions with the platform state, the model presented captures equilibria that is both insightful and representative of real-life dynamics.


Among the few papers that specifically consider SBDA matching markets, one that stands out is the recent work by \cite{kanoria2021facilitating}, which postulates a two-sided dynamic matching model with vertically-differentiated agents, and finds that platforms with unbalanced markets can improve welfare by forcing the short side to propose. There are several modelling choices distinguishing this work and mine, but I identify two main differences worth discussing. Firstly, the action space in \cite{kanoria2021facilitating} is far richer as it allows agents to both issue and receive proposals to the other side. Whilst this permits a focused study of platforms such as Bumble or Coffee Meets Bagel, with user interactions that permit this, the model does not adjust naturally to mechanisms such as the one in Tinder, where agents do not know ex-ante if the other agent has swiped on them and must factor this within the strategic cost/benefit analysis of deciding to `spend a swipe'. This is important since one of the main selling points of Tinder paid subscriptions is the ability to observe which users have already swiped right on you, thus providing a strategic advantage. Furthermore, the two key platform interventions studied, in line with their elaborate action space, involve restricting one side of the market from proposing or hiding information regarding the quality types of agents. On the other hand, this work explores mainly the impact of budget caps as a platform intervention in a way that is not directly modelled in the above, which is important given that this mechanism is more widely applicable to a broad range of SBDA's. Another exemplary model for SBDA markets is the one presented by \cite{immorlica2021designing}, which focuses on the problem of guiding the search process through type-contingent meeting rates for agents. This paper differs greatly with my work both in terms of its main research focus and on several modelling choices, but 


\begin{comment}
- Main differences

- Trait: Decentralised
- What it means: Matches result from the decision-making process of searching agents, with textbook examples involving agents who seek a job, a university, or a spouse. This differs from centralised matching context where matches are computed by a centralised authority who is fully informed of the preferences of all agents on both side of the market. Search models allow for, and characterize imperfect matches, but stable matches are still a nice theoretical benchmark
- Top works: \cite{gale_shapley_1962}
    - How they differ

- Trait: Two-sided
- What it means: Outcomes depend on the search process undertaken by both sides of the market; model aggregates tons of decision and information.

- Top works
    - What they found
    - How they differ

- Trait: Non-transferable utility
- What it means: Payoffs are not transferable amongst agents. This contrasts with PTU where a bargaining process must be undertaken to divide the within-match surplus.
- Top works: Roth and Sotomayor (1990).
    - What they found
    - How they differ

- Trait: heterogeneous preferences versus common ordinal preferences.



- Trait
- What it means
- Top works
    - What they found
    - How they differ


\begin{itemize}
    \item What is Tinder? (brief)
    \begin{itemize}
        \item When was it started?
        \item What is swiping?
        \item How popular it is?
    \end{itemize}
    \item Why does Tinder pose an interesting economic problem?
    \begin{itemize}
        \item Stage interaction
        \item Platform features: budgets, observability, directed search, asynchronicity
        \item Repeated games: curse of dimensionality, beliefs and meta-beliefs
    \end{itemize}
    \item What and how does this paper study?
    \begin{itemize}
        \item Model of two-sided search with strategic considerations
        \item Equilibrium refinement, computation, and analysis
        \item Planner considerations on directed search and budget setting
    \end{itemize}
    \item What does this paper contribute?
    \begin{itemize}
        \item First model to address budgeted search in Tinder?
        \item First model to combine idiosyncracy and pizzaz
        \item Case study for the use of computational techniques in 
    \end{itemize} 
    \item Searching and Matching
    \begin{itemize}
        \item \cite{gale_shapley_1962}, \cite{roth_sotomayor_1992}
        \item Two-sided: \cite{burdett1998two}, \cite{chade2006matching}, Smith, Adachi
        \item \textbf{Does not consider budgets}
        \begin{itemize}
            \item ... important as this is a way for planners to influence outcomes
        \end{itemize}
    \end{itemize}
    \item Mean-Field Game Theory: \cite{iyer2014mean}, \cite{gummadi2013optimal}, \cite{jovanovic1988anonymous}
    \begin{itemize}
        \item No models on MFG for Tinder
    \end{itemize} 
    \item Modern Dating Apps: \cite{olmeda2021towards}, \cite{kanoria2021facilitating}
    \begin{itemize}
        \item Not models where behaviour is derived from rational utility-maximizing assumptions 
    \end{itemize}
\end{itemize}
\end{comment}