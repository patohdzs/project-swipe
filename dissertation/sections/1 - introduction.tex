\section{Introduction}
\label{sec:section1}
It is widely acknowledged that the search for love is a complex social phenomenon, but in today's world, swipe-based dating platforms (SBDPs) seem to only make it trickier.
These platforms, exemplified by Tinder, Bumble, and Hinge, provide a gamified way of browsing through potential romantic partners by swiping through a stack of suggested candidates to indicate likes or dislikes for these, one profile at a time.  
In the search and matching literature, settings like these fall under the category of decentralised two-sided matching markets with search frictions \citep{kanoria2021facilitating} and, despite broad differences with traditional dating sites that perform centralised static matching, SBDPs have come to dominate the modern love market, with Tinder alone boasting 75 million monthly active users and 9.6 million paid subscribers as of 2021 \citep{web:tinder_stats}.

From a theoretical standpoint, search within SBDPs involves several complexities induced by platform-specific features, such as swiping caps, asynchronicity, and directed search algorithms.
These impose non-trivial constraints on the way utility-maximising agents strategise their search, but they have been sparsely studied in the economics literature due to the relative novelty of these platforms.
Overall, the prevalent role of SBDPs in shaping modern romantic interactions and their largely understudied nature motivates many different questions.
Nevertheless, exploring these requires a fundamental understanding of how users make decisions in these platforms: to put it simply, \textit{when should a utility-maximising user swipe right?}

This dissertation will explore the above within an SBDP setting, where agents with heterogeneous preferences on both sides of the market search sequentially for multiple romantic partners.
Crucially, I focus on explaining (what I refer to as) the `Fast-Swiping Men' puzzle: that is, the empirical observation that men in SBDPs respond with significantly higher swipe rates and face considerably worse matching outcomes than women. This phenomenon has been both a subject of empirical research \citep{tyson2016first} and a contentious discussion point within mainstream media \citep{web:vice_tindermen, web:wp_miserabletinder}, and yet a significant gap persists within the literature for an exploration of this through a theoretical lens\footnote{Among the surveyed literature, perhaps the only partial examination of this phenomenon is provided by \cite{kanoria2021facilitating}}. 
Such analysis would add significant value since the potential causes of this phenomenon (user patience, differential preferences, and strategic dominance) are all systematically endogenous with one another, thus demanding a rigorous model that can isolate these individual effects and trace their propagation across the SBDP market.
Fundamentally, I show how sex imbalances within the platform (which arise due to several exogenous factors) can explain the above disparities and, expanding on this, I outline a possible intervention under which differential swiping caps between sexes can alleviate inefficiencies stemming from asymmetric selectiveness in the market.

This work presents two main contributions to existing literature on the topic. 
Firstly, it constitutes one of a handful of attempts to model the market configurations arising within SBDPs, which is unsurprising due to the novelty of these platforms, but important given their current social relevance. 
Furthermore, it distinguishes itself from other similar works by directly considering the `Fast-Swiping Men' puzzle as well as the impact of swiping caps both as a constraint in the agent's search problem and a potential market correction mechanism.
Finally, this work provides an interesting case study for the use of computational methods within game theory (a field that has traditionally emphasised pure mathematical analysis), showing how the two approaches can be jointly applied to complicated questions, with computational methods providing quick explorations that can serve as an intuitive stepping stone towards formalising mathematical arguments.

The remainder of the paper is structured as follows. In \autoref{sec:section2}, I outline the theoretical framework for the model developed in this paper, and derive necessary conditions for both the platform steady state and agent best-responses. In \autoref{sec:section3}, I present a refined definition for the steady-state equilibrium of the model and perform computational comparative statics on several parameters, with the aim of explaining the `Fast-Swiping Men' phenomenon and a possible market intervention. In \autoref{sec:section4}, I utilise agent-based simulations to analyse the convergence and dynamics of my model, while \autoref{sec:section5} presents concluding remarks and outlines potential avenues for future research.

\subsection{Related Work}
\label{sec:section1.1}
The present work draws inspiration from two key branches of economics literature: that of search and matching theory, which studies the decision-making process of agents who seek, for example, a job, a business partner, or a spouse, and that of mean-field game theory, which models complex dynamic games involving a large number of players. 
I discuss each of these in turn, and then contrast this work with the handful of papers that have focused on specifically analysing SBDP markets.

Despite the abundance of papers within the search and matching literature, which has been amply surveyed by \cite{chade2017sorting}, I focus on works that consider the three defining features of SBDP markets: decentralised matching, non-transferable utility, and sequential search with frictions. 
A seminal paper at this intersection is that of \cite{burdett1997marriage}, which studies the marriage market for ex-ante heterogeneous agents under uniform random search, extending the work of \cite{becker1973theory} by showing that positive assortative matching can arise in a setting with search frictions.
Several extensions follow from this, considering idiosyncratic preferences \citep{burdett1998two}, noisy attractiveness observations \citep{chade2006matching}, and even convergence onto the set of stable matchings \citep{adachi2003search}. 
The framework outlined in this dissertation is perhaps most similar to that of \cite{burdett1998two}, with three major differences between the two. 
Firstly, the model developed in this paper extends the above by allowing for multiple partners within an agent's lifetime, a feature which was probably not significant within the labour market context considered by \cite{burdett1998two}, but which is nevertheless quintessential of SBDPs given their role in enabling casual relationships.
Furthermore, I extend the work of \citeauthor{burdett1998two} by allowing for sex-specific mass differences in the platform, as well as exogenous agent arrival flows; a point that was of noted interest for the authors themselves, and which is fundamental when considering the effects of sex imbalances within the platform. 
Finally, the model in this dissertation adopts a discrete time framework, departing from \cite{burdett1998two} and most of the recent matching literature. 
Although continuous-time models provide sharper analysis and more flexible empirical specifications \citep{burdett1999long}, this modelling choice lends itself naturally to agent-based simulations, which are used to explore equilibria convergence and dynamics in a richer manner.

On the other hand, mean-field game theory focuses on dynamic games with a large number of agents, for which curses of dimensionality arise due to intractable state spaces. 
Mean-field models tackle this issue by conditioning gameplay on the invariant state distribution amongst players, rather than tracking and considering the individual state of each opponent \citep{light2022mean}.
This simplifying assumption is cemented by a \textit{consistency check}, such that equilibria arise when rational play conditional on an aggregate state distribution maintains this as a fixed point. 
This approach, perhaps first considered by \cite{jovanovic1988anonymous} and \cite{hopenhayn1992entry}, has been successfully applied to settings such as network routing \citep{calderone2017markov}, dynamic auctions with learning \citep{iyer2014mean} and, perhaps most relevantly, online matching platforms \citep{kanoria2021facilitating,immorlica2021designing}.
In this paper, I rely on mean-field assumptions to abstract away from observability considerations: within SBDPs, the market history is unknown to players, therefore concepts such as Perfect Bayesian Equilibrium would require players to maintain and update beliefs over uncountable history spaces, and even beliefs over the beliefs of other players(a complication known as nested beliefs \citep{brandenburger1993hierarchies}).
This yields two central problems: first, that agent best-responses and equilibria become virtually impossible to compute, and, by extension, that the model assumes an unreasonable level of rationality on behalf of agents \citep{iyer2014mean}.
Thus, by conditioning interactions on the stationary platform state only, the model in this paper characterises equilibria that are both insightful and representative of real-life behaviour and dynamics. 

Among the few papers specifically considering SBDPs, \cite{kanoria2021facilitating} propose a dynamic two-sided model with vertically-differentiated agents, and show that platforms with unbalanced markets can improve welfare by forcing the short side to `propose' in all interactions. 
Furthermore, \cite{immorlica2021designing} focus on the problem of designing a directed search algorithm for SBDPs by endogenising type-contingent meeting rates for agents. 
Finally, \cite{arnosti2021managing} study the welfare losses that can arise in congested matching markets, and propose limiting the visibility of agents as a mechanism for overcoming this issue.
All of these papers present theoretical models with similar features, and these have largely influenced my work in several ways. 
Despite this, the above papers all model settings with one-to-one matchings only, thus differing fundamentally with my work in terms of the nature of endogeneity for agent departures. 
On one hand, this modelling choice makes their work and insights applicable to a wider variety of online matching platforms (such as AirBnb, TaskRabbit, etc.), but it also fails to capture an essential aspect specific to SBDPs that could have significant behavioural implications. 
Other than this (and some minor technical differences), the main point of distinction between my work and theirs is mostly one of perspective: whilst the above papers focus mostly on  platform mechanism design (by considering features such as information constraints or directed search algorithms), I instead seek to explain, from first-principles, how empirically-observed phenomena arises in SBDPs.