\section{Introduction}
\label{sec:Introduction} 
%To make a call to the introduction, put \ref{sec:Introduction} 

It is widely considered that the search for love is an intricate and complex social phenomenom, and in today's world, swipe-based dating applications seem to only make it trickier. 
These platforms, best exemplified by Tinder or Bumbble, provide a gameified way of browsing through potential romatic partners by swiping through a stack of suggestions to indicate likes or dislikes, one profile at a time. 
In the search and matching literature, these fall under the category of decentralized two-sided matching markets with online search \citep{kanoria2021facilitating}, emphasising the fact that \textit{a)} both sides of the market are comprised of rational agents, \textit{b)} matches only ocurr given a double coincidence of wants, and \textit{c)} romatic suggestions are presented in a \textit{sequential} manner to users. 
These apps thus differ widely from traditional dating sites where users are centrally and statically matched (such as ...), but have come to dominate the modern love market, with Tinder boasting XXX million users as of 2021 and YYY million paid subscribers.


From a theoretical standpoint, these platforms introduce many additional complexities that, due to their novelty, have been sparsely studied in the economics literature, with these falling into two categories: those arising from platform features and those arising from the intrinsic nature of the problem. 
The first of these refers to platform-specific factors such as suggestion algorithms, matching technologies, swiping caps, and asynchronicity, which are often determined exogenously and pose significant constraints to the way utility-maximizing agents strategize their search process. 
On the other hand, the generalized problem of search is inherently complex as it involves a dynamic game of incomplete information where, even though the stage interaction is simple, its repetion demands consideration of 

Overall, the prevalence of these swipe-based dating apps in modern social interactions, the theoretical complexities they induce, and the largely unstudied bla bla bla motivated this dissertation. Many different questions can be asked about these platforms, raging from the market configurations that emerge in equilibrium to the mechanism design considerations that induce social efficiency, but answers to all of these rest fundamentally on an understanding of how users make decisions in these platforms, or in other words, \textit{when should a user swipe right?}. This paper will explore the above by formulating a game-theoretic model of two-sided search within these platforms along with a corresponding definition of equilibrium. Using numerical methods, we 



\textbf{Points to discuss on introduction}
\begin{itemize}
    \item What is Tinder? (brief)
    \begin{itemize}
        \item When was it started?
        \item What is swiping?
        \item How popular it is?
    \end{itemize}
    \item Why does Tinder pose an interesting economic problem?
    \begin{itemize}
        \item Stage interaction
        \item Platform features: budgets, observability, directed search, asynchronicity
        \item Repeated games: curse of dimensionality, beliefs and meta-beliefs
    \end{itemize}
    \item What and how does this paper study?
    \begin{itemize}
        \item Model of two-sided search with strategic considerations
        \item Equilibrium refinement, computation, and analysis
        \item Planner considerations on directed search and budget setting
    \end{itemize}
    \item What does this paper contribute?
    \begin{itemize}
        \item First model to address budgeted search in Tinder?
        \item First model to combine idiosyncracy and pizzaz
        \item Case study for the use of computational techniques in 
    \end{itemize}
\end{itemize}
\subsection{Related Work}
\begin{itemize}
    \item Searching and Matching
    \begin{itemize}
        \item \cite{gale_shapley_1962}, \cite{roth_sotomayor_1992}
        \item Two-sided: \cite{burdett1998two}, \cite{chade2006matching}, Smith, Adachi
        \item \textbf{Does not consider budgets}
        \begin{itemize}
            \item ... important as this is a way for planners to influence outcomes
        \end{itemize}
    \end{itemize}
    \item Mean-Field Game Theory: \cite{iyer2014mean}, \cite{gummadi2013optimal}, \cite{jovanovic1988anonymous}
    \begin{itemize}
        \item No models on MFG for Tinder
    \end{itemize} 
    \item Modern Dating Apps: \cite{olmeda2021towards}, \cite{kanoria2021facilitating}
    \begin{itemize}
        \item Not models where behaviour is derived from rational utility-maximizing assumptions 
    \end{itemize}
\end{itemize}