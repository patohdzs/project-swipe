\section{Introduction}
\label{sec:section1}
It is widely acknowledged that the search for love is a complex social phenomenon, but in today's world, swipe-based dating applications (SBDA's) seem to only make it trickier.
These platforms, exemplified by Tinder, Bumble, or Hinge, provide a gamified way of browsing through potential romantic partners by swiping through a stack of suggested candidates to indicate likes or dislikes for these, one profile at a time. 
In the search and matching literature, settings like these fall under the category of decentralized two-sided matching markets with online search \citep{kanoria2021facilitating}, emphasising three main characteristics.
Firstly, that both sides of the market are comprised of decision-making agents undertaking a process of search.
Secondly, that matches occur as outcomes of independently-determined search decisions, rather than through a centralised algorithm. Thirdly, that romantic suggestions are presented in an online manner to users, stressing the importance of sequential rationality given that it is not possible to revert interactions with previous candidates.
These apps differ widely from traditional dating sites where users are centrally and statically matched (such as match.com or eHarmony), but have come to dominate the modern love market, with Tinder alone boasting 75 million monthly active users and 9.6 million paid subscribers as of 2021 \citep{web:tinder_stats}.

From a theoretical standpoint, search within SBDA's induces several additional complexities due to platform-specific features, such as swiping caps, asynchronicity, and directed search algorithms.
These impose non-trivial constraints on the way utility-maximising agents strategise their search process, but they have been sparsely studied in the economics literature due to the relative novelty of these platforms.
Overall, the prevalent role of SBDA's in shaping modern romantic interactions and their largely understudied nature motivates many different questions.
Nevertheless, exploring these demands a fundamental understanding of how users make decisions in these platforms: to put it simply, \textit{when should a utility-maximising user swipe right?}

This dissertation will explore the above within an SBDA platform setting, where agents with heterogeneous preferences on both sides of the market search simultaneously for multiple romantic partners.
Crucially, I focus on explaining (what I refer to as) the `Fast-Swiping Males' puzzle: that is, the empirical observation that men in SBDA's respond with significantly higher swipe rates and face considerably worse matching outcomes than women. This phenomenon has been both a subject of empirical research \citep{tyson2016first} and an extensively documented discussion point within mainstream media \citep{web:vice_tindermen, web:wp_miserabletinder} and yet, in spite of this, a significant gap persists within the literature for a discussion of this through a theoretical lens \footnote{Among the surveyed literature, perhaps the only partial examination of this phenomenon is provided by \cite{kanoria2021facilitating}}. 
This analysis is of vital importance given that the multiple factors that often get blamed for this phenomenon (user patience, differential preferences, and strategic dominance) are all systematically endogenous with one another, thus demanding a rigorous model that can isolate these individual effects and capture their propagation across the SBDA market.
Fundamentally, I argue that gender imbalances within the platform (which arise due to several exogenous factors) can explain swipe rate disparities within the platform and, expanding on this, I model a possible intervention where the swiping cap ratio between sexes can be set in a socially-efficient manner.

This work presents two main contributions to existing literature on the topic. 
Firstly, it constitutes one of a handful of attempts to model the market configurations arising within SBDA's, which is unsurprising due to the novelty of these platforms, but important given their current social relevance. 
Furthermore, this work distinguishes itself by considering the impact of swiping caps in the market, both as a constraint within the agent's search problem and a potential market correction mechanism.
Finally, this work provides an interesting case study for the use of computational methods within game theory, a field that has traditionally emphasised pure mathematical analysis. 
By pairing a rigorously-formulated model with numerical approximations and agent-based simulations, this dissertation exemplifies how the two approaches, rather than being mutually exclusive, can be jointly employed to explore complicated questions, as computational methods enable quick explorations that can serve as a stepping stone towards formalising mathematical arguments.

The remainder of the paper is structured as follows. In \autoref{sec:section2}, I outline the theoretical framework for the model developed in this paper, and derive necessary conditions for both the system steady state and the agent best-response correspondences. In \autoref{sec:section3}, I present a refined definition for the steady state equilibrium of the model and perform computational comparative statics on several parameters, with the aim of replicating stylized empirical facts and explaining the `Fast-Swiping Males' phenomenon. In \autoref{sec:section4}, I utilize agent-based simulation methods to analyse convergence and dynamics of my model, and present a discussion on socially-efficient budget interventions. Finally, \autoref{sec:section5} presents concluding remarks and outlines potential avenues for future research.

\subsection{Related Work}
The present work draws inspiration from two key branches of economics literature: that of search and matching, which studies the decision-making process of agents who seek, for example, a job, a business partner, or a spouse, and that of mean field game theory, which models complex dynamic games involving a large number of players. I discuss each of these in turn, and then contrast this work with the handful of papers that have focused on specifically analysing SBDA market configurations.

Despite the abundance of papers within the search and matching literature, which has been amply surveyed by \cite{chade2017sorting}, I draw focus on works that consider the three defining features of SBDA markets: decentralised matching, two-sidedness, and bilateral sequential interactions. A seminal paper at the heart of this intersection is that of \cite{burdett1997marriage}, which studies the marriage market for ex-ante heterogeneous agents under uniform random search, extending the work of \cite{becker1973theory} by showing that positive assortative matching can arise even in the absence of log-supermodularity.
Several extensions followed this work, considering settings with idiosyncratic preferences \citep{burdett1998two}, noisy attractiveness observations \citep{chade2006matching}, and even convergence onto the set of stable matchings \citep{adachi2003search}. These various different `flavours' of two-sided matching models served as great inspiration for this dissertation, and the framework developed hereafter is perhaps most similar to that of \cite{burdett1998two}, with three major distinctions. 
Firstly, the model formulated in this paper extends the above by allowing for multiple partners within a user's lifetime, a feature which was probably not significant within the labour market context considered by \cite{burdett1998two}, but which proves quintessential given the role of SBDA's in enabling casual relationships.
Furthermore, the model I present extends the above by allowing for sex-specific mass differences, as well as exogenous agent inflows; a point which was noted as a worthwhile extension by \citeauthor{burdett1998two} themselves, and which is fundamental in order to consider the effects of gender imbalances within the platform. 
Finally, the framework developed in this paper considers a discrete time framework, unlike \cite{burdett1998two} and most other works in the matching literature. Although continuous-time models provide sharper analysis and more flexible empirical specifications \citep{burdett1999long}, this modelling choice lends itself naturally to the use of agent-based simulations, which are employed to asses convergence and model dynamics in a richer manner. 

On the other hand, mean field game theory focuses on dynamic games with a large number of agents, for which curses of dimensionality often arise due to intractable state spaces \citep{maskin2001markov}.
To deal with this, mean field models consider individual interactions with the \textit{aggregate state} only, ie. the distributions over states and strategies within the game, rather than interactions with all other players.
This abstraction is coupled with the notion of a \textit{consistency check}, such that equilibrium arises when rational play given an aggregate state maintains this same state as a fixed point. 
The approach, first considered in the work of \cite{jovanovic1988anonymous}, greatly simplifies strategic settings with the aforementioned problem and has been successfully applied to settings such as network routing \citep{calderone2017markov} and dynamic auctions with learning \citep{iyer2014mean}.
In this paper, we rely on mean-field assumptions to abstract from observability considerations: within SBDA's, the market-wide history is unobservable to players, and thus traditional equilibrium concepts would demand beliefs over uncountable history spaces, and even beliefs over the beliefs other players may hold (a complication known as nested beliefs \citep{brandenburger1993hierarchies}).
This yields two central problems: first, that equilibria become virtually impossible to compute, and, by extension, that the model assumes an unreasonable level of rationality on behalf of agents.
Thus, by considering interactions with the platform state, the model presented in this paper characterizes equilibria that are both insightful and representative of real-life behaviour and dynamics.


Among the few papers that specifically consider SBDA matching markets, one that stands out is the recent work by \cite{kanoria2021facilitating}, which postulates a two-sided dynamic matching model with vertically-differentiated agents, and finds that platforms with unbalanced markets can improve welfare by forcing the short side to propose. Furthermore, the work presented by \cite{immorlica2021designing} focuses on the problem of determining a directed search algorithm in SBDA's through type-contingent meeting rates for agents. Both of these papers contain similar features within the theoretical models they develop, and these have largely influenced my work in several ways; for example, by embedding mean-field assumptions that simplify the SBDA market from a game theoretical perspective. Despite this, there are three main differences between my work and above that are worth discussing. %Firstly, the action space in \cite{kanoria2021facilitating} is far richer as it allows agents to both issue and receive proposals to the other side. Whilst this permits a focused study of platforms such as Bumble or Coffee Meets Bagel, with user interactions that permit this, the model does not adjust naturally to mechanisms such as the one in Tinder, where agents do not know ex-ante if the other agent has swiped on them and must factor this within the strategic cost/benefit analysis of deciding to `spend a swipe'. This is important since one of the main selling points of Tinder paid subscriptions is the ability to observe which users have already swiped right on you, thus providing a strategic advantage. Furthermore, the two key platform interventions studied, in line with their elaborate action space, involve restricting one side of the market from proposing or hiding information regarding the quality types of agents. 