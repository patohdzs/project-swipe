\section{Conclusion}
\label{sec:section5}
This paper studied the strategic behaviour of users in SBDP markets by formulating a model of two-sided search for agents with heterogeneous preferences and intertemporal action constraints. 
Using mean-field assumptions that hold for large markets, I provided an explicit characterisation of agent best-responses and used computational procedures to approximate SSE, exploring the effects of different exogenous parameters on both individual behaviour and the aggregate SBDP market.
Finally, I used ABM techniques to asses the convergence properties of my model, as well as its robustness under myopic best-response dynamics, to better asses not only if equilibria can be accurately computed, but also if they can be attained under relaxed gameplay conditions.
I focused particularly on explaining how the `Fast-Swiping Males' phenomenon can arise in unbalanced markets thanks to the endogenous relationship between an agent's patience, their swiping behaviour, and the market steady-state. 
Crucially, I identified that this puzzle is most likely the result of sex-specific arrival flow differences, although an interesting avenue of future research could study why these differences occur in the first place, perhaps by considering a relation with competing SBDPs and non-SBDP romantic search alternatives.


\subsection{Future Work}
There are several interesting avenues for future research concerning SBDPs. 
Firstly, although I provide a closed form characterisation of optimal stationary policies and prove the existence and uniqueness of SSE under balanced markets, future work could extend this by formalising these arguments for imbalanced markets.
Furthermore, numerous extensions could be made to study more focused aspects of SBDP markets. One modification of particular interest would involve a richer action set that allows for both casual and long-term matches (after which individuals leave the SBDP permanently), which could uncover interesting insights on the consequences of romantic goal incompatibility and how it could exacerbate inefficiencies in SBDPs. Finally, ABM could be used to test other properties of SSE, such as their stability under an evolutionary framework in a style similar to \cite{kanoria2021facilitating}.