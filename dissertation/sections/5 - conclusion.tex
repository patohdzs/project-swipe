\section{Conclusion}
\label{sec:section5}
This paper studied the strategic behaviour of users in SBDP markets by formulating a model of two-sided search for agents with heterogeneous preferences and intertemporal action constraints. 
Using mean-field assumptions that hold for large markets, I provided an explicit characterisation of agent best-responses and used computational procedures to approximate SSE, exploring the effects of different exogenous parameters on both individual behaviour and the aggregate SBDP market.
Finally, I used ABM techniques to asses the convergence properties of my model, as well as its robustness under myopic best-response dynamics, to determine if equilibria can be attained under relaxed gameplay conditions.
I placed particular focus on explaining how the `Fast-Swiping Men' phenomenon can arise in unbalanced markets due to the endogenous relation between an agent's patience, their swiping behaviour, and the market steady state. 
Crucially, I identified that this puzzle is most likely the result of exogenous differences in sex-specific arrival flows, which can be counterbalanced with opposite differences in their respective swiping caps. 
Nevertheless, an interesting direction for future research could involve studying why these inflow differences occur in the first place, perhaps by considering an endogenous relation with competing SBDPs and alternative romantic search markets. 

Overall, this work uncovers a number of actionable insights that could allow SBDPs to enhance both their user experience and their profitability.
The most direct application of these probably involves subscription pricing, since the main benefits that Tinder provides for its paid subscribers include: unlimited swipes, increased visibility, and the ability for an agent to observe profiles that have already liked them \citep{web:tinder_subscription}.
These benefits essentially equate to removing the three sources of search frictions explored by this paper: swiping constraints, market tightness, and strategic considerations, respectively.
As such, the conclusions outlined above could provide guidance to future research, for example, by motivating subscription pricing models that discriminate based on market tightness.
On a final note, one of the main benefits of the model presented in this paper is perhaps its flexibility, admitting to a number of potential extensions that could be used to study more focused aspects of SBDPs. 
One modification of particular interest would involve a richer action set that allows for both multiple casual matches and a single long-term match (after which agents leave the market permanently), which might uncover interesting insights on romantic incompatibility and associated search frictions in SBDPs.