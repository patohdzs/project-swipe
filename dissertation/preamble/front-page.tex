\pagestyle{empty} 
	\begin{center}
		\includegraphics[scale = 0.45]{preamble/warwick-crest.pdf}
		%Two images here for University of Warwick students, the colour crest and the black and white crest. Replace as appropriate!
	\end{center}
	\vspace{5mm}
	\begin{center}
		\textbf{\begin{LARGE}
		How Strong Is Your Tinder Game?\\
		\end{LARGE}}\textbf{\begin{LARGE}
		Two-Sided Search in Swipe-Based Dating Apps 
		\end{LARGE}}\
		\vspace{10mm}
	\end{center}
	\begin{center}
		\textbf{\large Patricio Hernandez Senosiain}
		\vspace{10mm}
	\end{center}
	
	\begin{abstract}
		In today's modern love market, swipe-based dating apps have a well-established presence, but platform features such as directed search algorithms and swiping caps add significant complexities to the user's search problem that have not been studied in existing literature.
		This paper formulates a game-theoretic model of two-sided search within swipe-based dating apps and, using numerical computation methods, approximates the steady-state equilibrium.
		The effects of various model parameters are assesed using comparative statics and replicate stylized facts observed in aggregate Tinder data.
		Finally,  the perspective of the planner to analyse how exogenously-determined platform features (such as the matching algorithm and the swiping caps) can be set to maximise welfare. 
		By analysing how platform design and its implied constraints affect user behaviour, this research aids dating platforms in improving social efficiency and provides a first step towards pricing models for subscriptions, a task of crucial importance considering that over 8 million people purchase subscriptions on Tinder alone. 
    \end{abstract}
    \vspace{20mm}
	\begin{center}
	     {\large Supervisor: Dr. Jonathan Cave}\\
		 \vspace{20mm}
		\textbf{\large Department of Economics}\\ 
		{\large 2019-2022\\}
	\end{center} 


