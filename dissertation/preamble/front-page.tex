\pagestyle{empty} 
	\begin{center}
		\includegraphics[scale = 0.45]{preamble/warwick-crest.pdf}
		%Two images here for University of Warwick students, the colour crest and the black and white crest. Replace as appropriate!
	\end{center}
	\vspace{5mm}
	\begin{center}
		\textbf{\begin{Large}
		How Strong Is Your Tinder Game?\\
		\end{Large}}\textbf{\begin{Large}
		Two-Sided Search in Swipe-Based Dating Apps 
		\end{Large}}\
		\vspace{10mm}
	\end{center}
	\begin{center}
		\textbf{\large Patricio Hernandez Senosiain}
		\vspace{10mm}
	\end{center}
	
	\begin{abstract}
		In today's love market, swipe-based dating apps have a well-established presence, but novel platform features can add significant complexities to the user's search problem  in ways that have not been studied in existing literature.
		This paper formulates a game-theoretic model of two-sided search within swipe-based dating apps and, using numerical computation methods, approximates the steady-state equilibrium.
		The effects of various model parameters are assesed using comparative statics and used to replicate and explain stylized facts observed in aggregate Tinder data.
		Finally, agent-based simulations are used to analyse off-path dynamics and discuss how exogenous platform features (such as the matching algorithm and the swiping caps) can be set in a socially-efficient manner. 
		By analysing the effects of platform design and its implied constraints on user behaviour, this research aids dating platforms in improving social efficiency and provides a first step towards pricing models for subscription plans to these platforms.
    \end{abstract}
    \vspace{20mm}
	\begin{center}
	     {\large Supervisor: Dr. Jonathan Cave}\\
		 \vspace{10mm}
		\textbf{\large Department of Economics}\\ 
		{\large 2019-2022\\}
	\end{center} 


