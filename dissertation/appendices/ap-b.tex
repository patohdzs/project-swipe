\section{Proofs for \autoref{sec:section2.3}}\label{appx: b} 
To simplify notation for \autoref{appx: b}, I denote the continuation value at budget $b$ by: 
\begin{equation*}
    \begin{aligned}
        &K_{b}:=\alpha \,\mathbb{E}_{\theta}\left[V_w(\theta', b)\right] 
    \end{aligned} 
\end{equation*}

\subsection{Proof for \autoref{prop:piecewiseV} and \autoref{cor:optpolicy}} 
\begin{proof}
    To prove \autoref{prop:piecewiseV}, fix any $b\in\mathcal{B}_w$. First, note that the difference between any two consecutive continuation values $K_b$ and $K_{b-1}$ must lie between 0 and $\overline{\mu}u(1)$. This is true since the value function denotes the expected lifetime sum of payoffs, and an additional right-swipe can provide an agent with, at most, an additional payoff of $\overline{\mu}u(1)$ and, at least, a payoff of $0$. Now consider:
    \begin{equation*}
        \begin{aligned} 
            V_w(\theta,b) \;&=\;\max\left\{\,\overline{\mu} \, u(\theta) +\alpha \,\mathbb{E}_\theta \Big[V_w(\theta', b-1)\Big]\,,\; \alpha\,\mathbb{E}_\theta \Big[ V_w(\theta', b)\Big]\,\right\}\\
            &=\; \max\left\{\,\overline{\mu} \, u(\theta) + K_{b-1} \,,\; K_b \,\right\}\\
            &=\; K_{b-1} + \max\left\{\,\overline{\mu} \, u(\theta) \,,\; K_b - K_{b-1}\,\right\}
        \end{aligned}
    \end{equation*}
    Since $u(\theta)$ is, by assumption, continuous and increasing over $\Theta$, then it must be that the second term on the RHS of the line above equals $K_b - K_{b-1}$ over $[0, \widetilde\omega_b]$, and $\overline{\mu} \, u(\theta)$ over $[\widetilde\omega_b, 1]$, where $\widetilde\omega_b$ satisfies: 
    \begin{equation*}
        \begin{aligned} 
            \overline\mu u(\widetilde\omega_b) &= K_b-K_{b-1} \\
            &=\alpha \, \mathbb{E}_\theta\Big[\,V_w(\theta',b)-V_w(\theta',b-1)\,\Big]
        \end{aligned} 
    \end{equation*}
    Thus, by considering the above function over the intervals $[0, \widetilde\omega_b]$ and $[\widetilde\omega_b, 1]$ separately, and substituting back the expressions for $K_b, K_{b-1}$, we conclude that:
    \begin{equation*}
    \begin{split}
        V_w(\theta,b)=\begin{cases}
            \overline\mu u(\theta) +\alpha \,\mathbb{E}_{\theta}\Big[V_w(\theta', b-1)\Big],& \theta> \widetilde \omega_b \\[10pt]
            \alpha \,\mathbb{E}_{\theta}\Big[V_w(\theta', b)\Big],& \theta\leq\widetilde \omega_b
        \end{cases} 
    \end{split}
    \end{equation*} 

    Furthermore, \autoref{cor:optpolicy} follows trivially from the above. 
\begin{comment} 
    When a woman with budget $b$ is presented a candidate with attractiveness $\theta \geq \widetilde\omega_b$ and she swipes right, her expected lifetime sum of discounted payoffs is:
    \begin{equation*}
        \begin{split}
            \overline\mu u(\theta) +\alpha \,\mathbb{E}_{\theta}\Big[V_w(\theta', b-1)\Big]\\
            = V_w(\theta,b)
        \end{split}
    \end{equation*}
    Alternatively, when presented a candidate with attractiveness $\theta<\widetilde\omega_b$ and she swipes left:
    \begin{equation*}
        \begin{split}
            \alpha \,\mathbb{E}_{\theta}\Big[V_w(\theta', b)\Big]\\
            = V_w(\theta,b)
        \end{split}
    \end{equation*} 
\end{comment}
\end{proof}

\subsection{Proof for \autoref{prop:recurrence relation}} 
\begin{proof}
To prove \autoref{prop:recurrence relation}, we rely on the two equations established by \autoref{prop:piecewiseV}; the first of which expresses the agent's value function in piecewise form, and the second of which describes the necessary condition for all reservation attractiveness values: 
\begin{align}
    \begin{split}\label{eq:A.1}
        V_w(\theta, b)=\begin{cases} 
            \overline\mu u(\theta) + \alpha \,\mathbb{E}_{\theta}\Big[V_w(\theta', b-1)\Big],& \theta> \widetilde \omega_b \\[10pt]
            \alpha \,\mathbb{E}_{\theta}\Big[V_w(\theta', b)\Big],& \theta\leq\widetilde \omega_b
        \end{cases}
    \end{split}\\[10pt]
    \begin{split}\label{eq:A.2}
        \overline\mu u(\widetilde\omega_b) = \alpha \, \mathbb{E}_\theta\Big[\,V_w(\theta',b)-V_w(\theta',b-1)\,\Big] 
    \end{split} 
\end{align}  

Starting out with \autoref{eq:A.2} and expanding out the expectation operator, we can use \ref{eq:A.1} to substitute in the piecewise definitions of $V_w(\theta,b)$ over the appropriate intervals: 
\begin{equation}\label{eq:A.3}
    \begin{split}
        \overline\mu u(\widetilde\omega_b) &= \alpha \,\int^1_0 V_w(\theta',b)-V_w(\theta',b-1)\,dF_m(\theta')\\
                                           &=\alpha \int^{\widetilde\omega_b}_0 K_b\,dF_m(\theta') \;+\; \alpha \int^1_{\widetilde\omega_b}\,\overline\mu u(\theta') + K_{b-1}\,dF_m(\theta')\\ 
                                           & \quad -\,\alpha \int^{\widetilde\omega_{b-1}}_0 K_{b-1}\,dF_m(\theta') \;-\; \alpha \int^1_{\widetilde\omega_{b-1}} \overline\mu u(\theta') + K_{b-2}\,dF_m(\theta')
    \end{split}
\end{equation}

Furthermore, Equation \ref{eq:A.2} implies that:

$$
\overline\mu u(\widetilde\omega_b) +K_{b-1}= K_b
$$

$$
\overline\mu u(\widetilde\omega_{b-1}) +K_{b-2}=K_{b-1}
$$

Then, by substituting these expressions into \ref{eq:A.3}, we arrive at \ref{eq:A.4}: 
\begin{equation}\label{eq:A.4}
    \begin{split}
        \overline\mu u(\widetilde\omega_b) &=\alpha \int^{\widetilde\omega_b}_0 \overline\mu u(\widetilde\omega_b) +K_{b-1}\,dF_m(\theta') \;+\; \alpha \int^1_{\widetilde\omega_b} \,\overline\mu u(\theta') + K_{b-1}\,dF_m(\theta')\\ 
                                           & \quad -\,\alpha \int^{\widetilde\omega_{b-1}}_0 K_{b-1}\,dF_m(\theta') \;-\; \alpha \int^1_{\widetilde\omega_{b-1}} \overline\mu u(\theta') + K_{b-1}-\overline\mu u(\widetilde\omega_{b-1})\,dF_m(\theta')
    \end{split}
\end{equation}

With some algebra, this simplifies down to the recurrence relation in \autoref{eq:recurrence relation}:  
\begin{equation}
    u(\widetilde\omega_b)=\alpha   u(\widetilde\omega_b)F_m(\widetilde\omega_b) \;+\; \alpha  u(\widetilde\omega_{b-1})\Big[1  - F_m(\widetilde\omega_{b-1})\Big] \;+\; \alpha\int^{\widetilde\omega_{b-1}}_{\widetilde\omega_b} u(\theta') \,dF_m(\theta') 
\end{equation}

Furthermore, to obtain the initial condition for the above, note that the right-swiping budget constraint imposes $V_w(\theta,0)=0, \forall b\in \mathcal{B}_w$. Then, \ref{eq:A.1} and \ref{eq:A.2} simplify to: 
\begin{align}
    \begin{split}\label{eq:A.5} 
        V_w(\theta, 1)=\begin{cases} 
            \overline\mu u(\theta),& \theta> \widetilde \omega_1 \\[6pt] 
            \alpha \,\mathbb{E}_{\theta}\Big[V_w(\theta', 1)\Big],& \theta\leq\widetilde \omega_1
        \end{cases}
    \end{split}\\[10pt]
    \begin{split}\label{eq:A.6}
        \overline\mu u(\widetilde\omega_1) = \alpha \, \mathbb{E}_\theta\Big[\,V_w(\theta',1)\,\Big] 
    \end{split} 
\end{align}

Beginning with \ref{eq:A.6}, we simplify until arriving at \autoref{eq:initial condition}: 
\begin{equation*} 
    \begin{split}
        \overline\mu u(\widetilde\omega_1) &= \alpha \, \mathbb{E}_\theta\Big[\,V_w(\theta',1)\,\Big]\\
        &= \alpha \,\int^{\widetilde\omega_1}_0\,K_1\,dF_m(\theta') + \alpha \,\int_{\widetilde\omega_1}^1 \overline\mu u(\theta')\,dF_m(\theta')\\
        &= \alpha \,\int^{\widetilde\omega_1}_0 \overline\mu u(\widetilde\omega_1)\,dF_m(\theta') + \alpha \,\int_{\widetilde\omega_1}^1 \overline\mu u(\theta')\,dF_m(\theta')\\
        &= \alpha \overline\mu u(\widetilde\omega_1)F_m(\widetilde\omega_1) + \alpha \,\int_{\widetilde\omega_1}^1 \overline\mu u(\theta')\,dF_m(\theta')\\
        \implies u(\widetilde\omega_1) &= \alpha u(\widetilde\omega_1)F_m(\widetilde\omega_1) + \alpha \,\int_{\widetilde\omega_1}^1 u(\theta')\,dF_m(\theta') 
    \end{split}
\end{equation*}  

To conclude the proof, note that the existence and uniqueness of some $\widetilde\omega_b$ that satisfies \ref{eq:A.2} is guaranteed given the assumptions on $u(\theta)$ being continuous and strictly increasing. Since the difference between any two consecutive continuation values must lie strictly between 0 and $\overline{\mu}u(1)$, then, by the Intermediate Value Theorem, there exists one and only one root $\widetilde\omega_b$ satisfying \ref{eq:A.2} and, by extension the above reoccurrence relation. 
\end{proof}